\usepackage[usenames,dvipsnames]{color}
\definecolor{mycitegreen}{RGB}{0,90,0}
\definecolor{myrefblue}{RGB}{90,0,0}
\definecolor{cyred}{RGB}{140,0,0}
\definecolor{commentgreen}{RGB}{0,80,0}
\usepackage[draft=true, final=true, colorlinks=true, citecolor=mycitegreen, linkcolor=myrefblue, anchorcolor=myrefblue, pagebackref=false]{hyperref}
\usepackage{graphics,enumitem}
\usepackage{epsfig}
\usepackage[captionskip=0pt, font={small}]{subfig}
\usepackage{color}
\usepackage{url}
\usepackage{xspace}
\let\proof\relax
\let\endproof\relax
\usepackage{amsthm}
%\usepackage[charter]{template/mathdesign/mathdesign}
%\usepackage{MinionPro}
%\setlength{\textheight}{9.4in}
%setlength{\textheight}{9.25in}
%\setlength{\textwidth}{6.75in}
%\setlength{\columnsep}{.33in}
%\usepackage[]{geometry}

%\usepackage{algpseudocode}
%\usepackage{rawfonts}
%\include{amsthm.sty}
% Define new commands
%\usepackage{amsthm}
%\usepackage{fmtcount}
%\usepackage{threeparttable}

\iffalse

%% PACKAGE %%
%\usepackage[font=small,labelfont=bf,tableposition=top]{caption}
%\usepackage[font=footnotesize]{subfig}
%\usepackage{caption}
\usepackage{floatrow}
% Table float box with bottom caption, box width adjusted to content
\newfloatcommand{capbtabbox}{table}[][\FBwidth]
\usepackage{blindtext}

%\usepackage{sectsty}
\usepackage{graphicx}
%\usepackage{times}
\usepackage{epsfig}
\usepackage{epstopdf}
\usepackage{subfigure}
\usepackage{url}
\def\UrlBreaks{\do\A\do\B\do\C\do\D\do\E\do\F\do\G\do\H\do\I\do\J\do\K\do\L\do\M\do\N\do\O\do\P\do\Q\do\R\do\S\do\T\do\U\do\V\do\W\do\X\do\Y\do\Z\do\[\do\\\do\]\do\^\do\_\do\`\do\a\do\b\do\c\do\d\do\e\do\f\do\g\do\h\do\i\do\j\do\k\do\l\do\m\do\n\do\o\do\p\do\q\do\r\do\s\do\t\do\u\do\v\do\w\do\x\do\y\do\z\do\0\do\1\do\2\do\3\do\4\do\5\do\6\do\7\do\8\do\9\do\.\do\@\do\\\do\/\do\!\do\_\do\|\do\;\do\>\do\]\do\)\do\,\do\?\do\'\do+\do\=\do\#}%
\usepackage{cite}%
\usepackage{color}
\usepackage[normalem]{ulem}
\usepackage{fmtcount}
\usepackage{times}
\usepackage{cite}
\usepackage{amsfonts,amssymb}
\usepackage{balance}
\usepackage{verbatim}
\usepackage{appendix}
%\usepackage{algorithmicx,array}
\usepackage{bbding}
\usepackage{floatrow}
\usepackage{tabularx}
\fi
\usepackage[english]{babel}
\usepackage{algorithm}
\usepackage{algorithmic}
\usepackage{balance}
\usepackage{multirow}

%\usepackage{tikz}
%\usepackage{algorithm}
%\usepackage{algorithmic,eqparbox,array}
\renewcommand\algorithmiccomment[1]{%
    \hfill\///\ \eqparbox{COMMENT}{#1}%
}
\newcommand\LONGCOMMENT[1]{%
    \hfill\///\ \begin{minipage}[t]{\eqboxwidth{COMMENT}}#1\strut\end{minipage}%
}

\usepackage{tikz}
\newcommand*\circled[1]{\tikz[baseline=(char.base)]{
            \node[shape=circle,draw,inner sep=0pt] (char) {#1};}}

% use color in table
\usepackage{xcolor,colortbl}
\newcommand{\mc}[2]{\multicolumn{#1}{c}{#2}}
\definecolor{Gray}{gray}{0.85}
\definecolor{LightCyan}{rgb}{0.88,1,1}
\newcolumntype{a}{>{\columncolor{Gray}}c}
\newcolumntype{b}{>{\columncolor{white}}c}

% Compact itemize and enumerate.  Note that they use the same counters and
% symbols as the usual itemize and enumerate environments.
\def\compactify{\itemsep=0pt \topsep=0pt \partopsep=0pt \parsep=0pt}
 \let\latexusecounter=\usecounter
 \newenvironment{CompactItemize}
   {\def\usecounter{\compactify\latexusecounter}
    \begin{itemize}}
   {\end{itemize}\let\usecounter=\latexusecounter}
 \newenvironment{CompactEnumerate}
   {\def\usecounter{\compactify\latexusecounter}
    \begin{enumerate}}
   {\end{enumerate}\let\usecounter=\latexusecounter}


\newenvironment{noindlist}
 {\begin{list}{*}
 {\leftmargin=0.35em \itemindent=0em \labelwidth = 0em \labelsep = 0.1em \itemsep = 0em \parsep = 0em \topsep = 0em \listparindent = 0.5in}}
 {\end{list}}

\newenvironment{noinditemize}{
\begin{itemize}
  \setlength{\itemsep}{0pt}
  \setlength{\parskip}{-3pt}
  \setlength{\parsep}{-3pt}
}{\end{itemize}}





%% ----------------Customized commands -----------------%
%
\newtheorem{definition}{Definition}
\newtheorem{lemma}{Lemma}
\newtheorem{prop}{Proposition}
\newtheorem{theorem}[lemma]{Theorem}
%\renewcommand{\paragraph}[1]{\smallskip \noindent \textbf{#1}\qquad}


% Paragraph Highlight
\newcommand{\paragraphb}[1]{\vspace{1mm}\noindent{\bf #1.} \quad}
\newcommand{\paragraphe}[1]{\vspace{1mm}\noindent{\em #1.} \quad}
\newcommand{\paragraphbe}[1]{\vspace{1mm}\noindent{\bf \em #1.} \quad}


% Customized shorts
\def\ie{\textit{i.e.}\xspace}
\def\etal{\textit{et al.}\xspace}
\def\etc{\textit{etc.}\xspace}
\def\eg{\textit{e.g.}\xspace}
\def\wrt{\textit{w.r.t.}\xspace}
\def\name{\textsc{InFrame}\xspace}

\newcommand{\SWITCH}[1]{\State \textbf{switch} (#1)}
\newcommand{\ENDSWITCH}{\State \textbf{end switch}}
\newcommand{\CASE}[1]{\State \textbf{case} #1\textbf{:} \begin{ALC@g}}
\newcommand{\ENDCASE}{\end{ALC@g}}
\newcommand{\CASELINE}[1]{\State \textbf{case} #1\textbf{:} }
\newcommand{\DEFAULT}{\State \textbf{default:} \begin{ALC@g}}
\newcommand{\ENDDEFAULT}{\end{ALC@g}}
\newcommand{\DEFAULTLINE}[1]{\State \textbf{default:} }


\newcommand{\interfigskip}{\vspace{-12pt}}
\newcommand{\listskip}{\vspace{-5pt}}
\newcommand{\listskips}{\vspace{-3pt}}

% Comment Colors 
\iffalse
%\algnewcommand\algorithmicswitch{\textbf{switch}}
%\algnewcommand\algorithmiccase{\textbf{case}}


%%\algnewcommand\algorithmicassert{\texttt{assert}}
%%\algnewcommand\Assert[1]{\State \algorithmicassert(#1)}%
%% New "environments"
%\algdef{SE}[SWITCH]{Switch}{EndSwitch}[1]{\algorithmicswitch\ #1\ \algorithmicdo}{\algorithmicend\ \algorithmicswitch}%
%\algdef{SE}[CASE]{Case}{EndCase}[1]{\algorithmiccase\ #1}{\algorithmicend\ \algorithmiccase}%
%\algtext*{EndSwitch}
%\algtext*{EndCase}%

% Comment colors
%\newcommand{\comment}{\textcolor{black}}
%\newcommand{\prooffontsize}{\fontsize{8pt}{9.3pt}\selectfont}
%\newcommand{\cready}{\textcolor{black}}
\fi


%\newif\ifdebugdoc\debugdocfalse
\newif\ifdebugdoc\debugdoctrue


\ifdebugdoc
%% Writing Mode
\newcommand{\fhl}[1]{\textcolor{blue}{#1}}
\newcommand{\fyi}[1]{}
%\newcommand{\fyi}[1]{\footnote{\textcolor{blue}{fyi:#1}}}
\newcommand{\remind}[1]{\footnote{\textit{[\uwave{Remind:} #1]}}}
\newcommand{\repl}[2]{\textcolor{red}{#1}\textcolor{blue}{\sout{#2}}}
\newcommand{\add}[1]{\textcolor{red}{#1}}
\newcommand{\del}[1]{\textcolor{blue}{\sout{#1}}}
\newcommand{\outline}[1]{\textbf{\colorbox{yellow}{Outline:}\textcolor{red}{#1.}}}
\newcommand{\old}[1]{\large{\colorbox{blue}{Former: #1}}}
%\newcommand{\chunyi}[1]{\footnote{\colorbox{yellow}{Chunyi:} #1.}}
\newcommand{\chunyi}[1]{\textcolor{blue}{Chunyi:#1}}
\newcommand{\anran}[1]{\footnote{\colorbox{red}{Anran:} #1.}}
\newcommand{\AnranAdd}[1]{\textcolor[rgb]{0.5,0.5,0}{#1}}
\newcommand{\AnranDel}[1]{\sout{#1}}


\newcommand{\frepl}[2]{\textcolor{red}{#1}\textcolor{blue}{\sout{#2}}}
\newcommand{\fadd}[1]{\textcolor{red}{#1}}
\newcommand{\fdel}[1]{\textcolor{blue}{\sout{#1}}}
\newcommand{\todo}[1]{\textcolor{red}{\textbf{ToDo: #1}}}

%\newcommand{\comment}{\textcolor{black}}
\newcommand{\prooffontsize}{\fontsize{8pt}{9.3pt}\selectfont}
\newcommand{\cready}{\textcolor{black}}
\newcommand{\newcomment}{\textcolor{black}}

%\newcommand{\cy}{\textcolor{cyred}{[CY-FIXME]}~\textcolor{cyred}}
%\newcommand{\cycomment}{\textcolor{cyred}}
\newcommand{\cycomment}{\textcolor{black}}
%\newcommand{\matt}[1]{{{#1}}} 
%\newcommand{\matt}[1]{{\color{blue}{#1}}}
\newcommand{\matt}[1]{{\color{black}{#1}}}
%\newcommand{\mattc}[1]{{\color{blue}\bf\em{}}} 
%\newcommand{\mattc}[1]{{\color{blue}\bf\em{[matt: #1]}}} 
\newcommand{\codecomment}{\textcolor{commentgreen}}
%\newcommand{\pbg}{\textcolor{green}}



%\newcommand{\corule}{Rule\xspace}
%\newcommand{\orule}{rule\xspace}
%\newcommand{\orules}{rules\xspace}
%
%\newcommand{\cprops}{Service properties\xspace}
%\newcommand{\props}{service properties\xspace}
%
%
%% To make the FIXMEs go away, comment out this line...
%\newcommand{\fixme}[1]{{\bf\textcolor{red}{[#1]}}}
%% ...and uncomment this one.
%%\newcommand{\fixme}[1]{}
%\newcommand{\helpme}[1]{{\bf\textcolor{red}{#1}}}


\else
%%Submission Mode
\newcommand{\fhl}[1]{#1}
\newcommand{\fyi}[1]{}
\newcommand{\remind}[1]{#1}
\newcommand{\repl}[2]{#1}
\newcommand{\add}[1]{#1}
\newcommand{\del}[1]{}
\newcommand{\chunyi}[1]{}
\newcommand{\outline}[1]{}
\newcommand{\frepl}[2]{#1}
\newcommand{\fadd}[1]{#1}

\newcommand{\fdel}[1]{}  %due to space limitation
\newcommand{\todo}[1]{}
%\newcommand{\frepl}[2]{\textcolor{red}{#1}\textcolor{blue}{\sout{#2}}}
%\newcommand{\fadd}[1]{\textcolor{red}{#1}}
%\newcommand{\fdel}[1]{\textcolor{blue}{\sout{#1}}}
\fi

%\newif\ifflow\flowfalse
\newif\ifflow\flowtrue
\ifflow
\newcommand{\p}[1]{\vskip 1ex\noindent\colorbox{yellow}{\parbox{\columnwidth}{\textbf{Point:} #1}}}
\newcommand{\q}[1]{\vskip 1ex\noindent\colorbox{cyan}{\parbox{\columnwidth}{\textbf{Question:} #1}}}
\else
\newcommand{\p}[1]{}
\newcommand{\q}[1]{}
\fi

%\newcommand{\fadd}[1]{\textcolor{red}{#1}}
%\newcommand{\fdel}[1]{\textcolor{blue}{\sout{#1}}}


\renewcommand{\algorithmicrequire}{\textbf{Input:}}
\renewcommand{\algorithmicensure}{\textbf{Output:}}


%\newcommand{\fdel}[1]{}
%\newcommand{\fadd}[1]{#1}f

%\baselineskip=12.1bp
%\showthe\baselineskip

\setlength{\pdfpagewidth}{8.5in}
\setlength{\pdfpageheight}{11in}

